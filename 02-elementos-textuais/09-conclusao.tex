% -----------------------------------------------------------------------------
% Conclusão
% -----------------------------------------------------------------------------

\chapter{Conclusão}
\label{chap:conclusao}

Este trabalho realizou um estudo a respeito da viabilidade de implementação de um \textit{design system} em uma empresa de tecnologia de médio porte. A metodologia adotada durante a avaliação consistiu em uma fase de entrevistas, onde desejava-se entender se os colaboradores da organização ansiavam por uma ferramenta de tal natureza, além de uma fase de implementação, onde foi desenvolvido um protótipo do sistema desejado.

Os resultados obtidos durante a etapa de entrevistas estão completamente alinhados com o que foi observado nos estudos de trabalhos relacionados. Problemas como inconsistências de usabilidade do produto, lento ciclo de desenvolvimento e a falta de diretrizes claras e objetivas que guiam a tomada de decisões de design, são fatores comuns que prejudicam a capacidade de escalabilidade dos produtos a longo prazo. Além disso, foi constatado que, conforme uma organização cresce em termos de número de funcionários, mais difícil se torna a gestão de comunicação e do conhecimento interno. Nesse contexto, um \textit{design system} cumpre o papel fundamental de centralização do conhecimento e diminuição da complexidade social de uma empresa.

Durante a fase de implementação, por sua vez, ficou clara a maneira como a existência de uma biblioteca de componentes bem estrutura impacta positivamente a velocidade de desenvolvimento e o nível de consistência de um produto. Ressalta-se, entretanto, a preocupação de não se criar dependência alguma entre a biblioteca de componentes e \textit{frameworks} ou tecnologias específicas, uma vez que isso contribui para o fenômeno de defasamento da ferramenta.

Também foi percebido que a criação de um \textit{design system} completo não é uma tarefa fácil de ser realizada. Para tanto, faz-se necessário o envolvimento e participação de profissionais das mais diversas áreas de atuação, uma vez que a ferramenta se propõe a atender às necessidades da organização como um todo. O maior desafio, portanto, seria manter a ferramenta constantemente atualizada, garantindo que a mesma não caia em desuso.

Finalmente, como sugestão de trabalho futuro é proposta a criação de uma nova biblioteca de componentes, dessa vez utilizando a tecnologia conhecida como \textit{Web Componentes}, baseada em um dos produtos mais recentes da empresa avaliada no experimento do presente trabalho. Tal produto já possui uma identidade visual mais moderna e bem padronizada. Em sequência, as próximas atividades a serem executadas seriam a definição dos artefatos de Tom de Voz e Princípios de Design, concluindo assim um \textit{design system} completo que transformará a maneira que a plataforma da empresa é gerenciada.
