% -----------------------------------------------------------------------------
% Entrevistas com Stakeholders
% -----------------------------------------------------------------------------

\chapter{Entrevistas com \textit{Stakeholders}}
\label{chap:entrevistas}


Como forma de validar a real necessidade de criação de um \textit{design system} na empresa avaliada no experimento, foram realizadas entrevistas com os principais \textit{stakeholders} do projeto: designers e desenvolvedores \textit{frontend}. Neste capítulo são apresentados, de maneira consolidada, os resultados obtidos como consequência de tais entrevistas.

Para garantir que os resultados não fossem enviezados em um único contexto da organização, foram selecionados 3 profissionais com desafios destintos dentro da empresa, como representantes de cada um dos grupos.

Todas as entrevistas foram gravadas, conforme conscentimento dos participantes. Para garantir o anônimato dos envolvidos, todos os dados apresentados neste capítulo estarão associados a nomes de marcação, não expondo nomes e dados pessoais reais dos voluntários.

\section{Designers}
\label{sec:entrevistasDesigners}

De acordo com o roteiro de entrevista ilustrado no Apêndice A deste projeto, os designers avaliados foram submetidos a algumas perguntas genéricas onde puderam expor um pouco dos detalhes do seu dia-a-dia de trabalho. O \autoref{table:designersResearch} apresenta as características de cada um dos profissionais entrevistados.

\begin{quadro}
\centering
\begin{tabular}{|m{4cm}|m{10cm}|} \hline
	
	\multicolumn{1}{|c|}{\bfseries Voluntário} & \multicolumn{1}{c|}{\bfseries Contexto} \\\hline
	
	 Designer 1 & Profissional mais experiente do grupo, sendo a principal referência de design em toda a organização. Está diretamente envolvido em um projeto mais novo e bastante promissor da empresa. Nesse projeto, teve a liberdade de definir novos padrões de design que hoje são tomados como referência para todos os novos produtos da empresa. \\\hline
	 
	 Designer 2 & Pilar de design do maior \textit{squad} da empresa, que tem como objetivo dar manutenção e evoluir todo o fluxo de comunicação da plataforma. Está diretamente envolvido com sistemas legados do produto, e tem como principal desafio garantir uma boa experiência do usuário ao longo do uso da plataforma. \\\hline
	 
	 Designer 3 &  Responsável pelo design da área de marketing da organização, estando diretamente envolvido em campanhas de eventos, contratação e lançamento de novos produtos. Também é responsável pelas artes do site e do blog de tecnologia da empresa. \\\hline
    
\end{tabular}
\caption{Características dos designers entrevistados}
\fonte{Próprio autor}
\label{table:designersResearch}
\end{quadro}

A seguir, são apresentados os resultados consolidados de perguntas chave realizadas nas entrevistas.

\textbf{Questão 1:} Contextualize, brevemente, a maneira como os processos de design são adotados na empresa, atualmente.

\begin{quote}
    A maioria das empresas de tecnologia são focadas no desenvolvedor. Por isso, grande parte dos processos de desenvolvimento existentes estão associados ao código-fonte, sendo o design muitas vezes negligenciado. Por muito tempo essa foi a realidade da Dito, porém nos últimos 2 anos a área de design começou a conquistar seu espaço na organização, tendo, pela primeira vez, profissionais especializados da área designados a evoluírem a plataforma com foco no usuário final do produto.
    
    Foi a partir desse momento que iniciaram-se trabalhos para definir os primeiros processos de design da empresa. O primeiro deles foi a inclusão de ferramentas como Sketch e Zeplin na rotina dos designers e desenvolvedores \textit{frontend}. Com elas, o designer exerce seu papel de pensar na melhor interface de usuário possível para um dado problema e, após sua prototipação, o desenvolvedor tem acesso a arte de maneira detalhada tecnicamente. Isso fez com que as novas interfaces desenvolvidas, desde então, fossem mais fidedignas àquilo idezaliado pelo designer.
    
    Apesar do considerável avanço, ainda entende-se que a empresa está "engatinhando" no que se diz respeito à processos de design. Ainda existe uma visão geral de que design se restringe unicamente à criação de interfaces. No que se tange à concepção de produtos, não está bem definido um processo de design escalável. Pensando nisso, um dos \textit{squads} da empresa vem tentando implacar um novo modelo conhecido como \textit{Double Diamond Design}, proposto pela metodologia de \textit{design thinking}. Tal processo é dividido em fases de imersão, ideação, prototipação e testes de usabilidade, objetivando a construção de produtos de maneira mais rápida e aderente às necessidades do usuários final.
    
    Na área de marketing, por sua vez, o contexto é um pouco diferente: contando apenas com dois profissionais alocadas para gerenciar toda a área, a criação de uma \textit{styleguide} se tornou uma necessidade latente. A maneira encontrada para se ganhar velocidade, e atender à toda a demanda da empresa, foi a criação de templates. Tal \textit{styleguide} foi criada a partir de um dos produtos da plataforma, porém não existe ligação alguma entre os dois artefatos.
\end{quote}

\textbf{Questão 2:} Quais as maiores dificuldades e frustrações que você encontra no seu trabalho, como consequência dos processos atuais de design?

\begin{quote}
    Tendo em vista que a maturidade em design da Dito está passando por um processo de evolução, os designers da empresa ainda precisam empregar uma quantidade de esforço significativa para a construção de novos protótipos de interfaces de usuários, uma vez que a capacidade de reusabilidade de componentes ainda está aquém do ideal. 
    
    Por conta disso, as maiores frustrações do time, em geral, estão relacionadas à impossibilidade de se ter um contato mais próximo com o usuário final do produto. Deseja-se que tarefas de teste de usabilidade e metrificação de maturidade façam parte do processo de design da Dito, porém atualmente isso é inviável pois os designers estão muito ocupados em dar vazão à demandas de criação de novas interfaces.
    
    Também destaca-se a falta de uma figura exclusivamente responsável por definir e garantir que as diretrizes de design estão sendo seguidas pela organização como um todo. Espera-se que essa pessoa também exerça o papel de tutor de talentos menos experientes, papel este inexistente na conjuntura atual da empresa.
\end{quote}

\textbf{Questão 3:} Como são tomadas decisões de design na empresa? E no seu time? Quem são as pessoas envolvidas?

\begin{quote}
    Basicamente, a partir da apresentação de um dado desafio, inicia-se uma fase de análise aprofundada do problema. Durante essa fase, é bastante comum o envolvimento de profissionais de negócio e desenvolvedores para a validação da viabilidade de possíveis soluções. Vez ou outra acontecem alterações nas decisões de design pré-estabelecidas devido à inviabilidade técnica da proposta. Portanto, esse tipo de tomada de decisão já é fruto de um trabalho colaborativo entre diferentes perfis de profissionais.
    
    Entretanto, o núcleo das discussões normalmente acontece no contexto individual do \textit{squad} responsável pelo problema. Não existe consenso geral de diretrizes a serem seguidas pelos times. Dessa forma, o cenário atual é caracterizado pelo trabalho isolado dos designers da empresa. Cada um deles se mantém focado exclusivamente nos desafios do seu próprio \textit{squad}, não sendo comum um intercâmbio de experiências entre os membros da área de design como um todo.
    
    Foram levantadas, ainda, situações em que decisões de design foram tomadas por profissionais de negócio, mesmo quando em desacordo do ponto de vista do designer responsável pela entrega. Devido à falta de autoridade da área, sendo um claro reflexo da imaturidade da mesma, a experiência do usuário acaba sendo prejudicada em detrimento do cumprimento de prazos estabelecidos para entregas.
\end{quote}

\textbf{Questão 4:} Como você classifica a velocidade de desenvolvimento de novas interfaces de usuários atualmente? Por que?

\begin{quote}
    De maneira geral, está mais rápida do que o observado em momentos anteriores à definição dos processos inicias de design, porém poderia estar melhor caso existisse uma biblioteca de componentes bem estruturada.
\end{quote}

\textbf{Questão 5:} Você acredita que um \textit{design system} poderia ser uma opção plausível para alavancar a capacidade de escalabilidade do produto da empresa?

\begin{quote}
    De maneira unânime, sim.

    A padronização de ferramentas e processos é, sem dúvidas, um excelente promotor do potencial de escalabilidade do produto. Estima-se que, com uma biblioteca de componentes bem estruturada, tarefas de prototipação seriam executadas 4 vezes mais rapidamente.
    
    Entende-se, entretanto, que a criação de um \textit{design system} é um projeto bastante ambicioso e complexo, fazendo-se necessário o envolvimento, preferencialmente exclusivo, de uma série de profissionais com nível avançado de senioridade. O resultado final do projeto é um artefato que tem a pretenção de resolver problemas de comunicação de toda uma organização, portanto sua concepção deve ser embasada nas dores de todos os grupos que compõem a empresa.
    
    Ressalta-se ainda que, o processo de criação de um \textit{design system} deve ser encarado como um produto e não um projeto. Para tanto, o mesmo deve estar em constante evolução, mantendo sua documentação viva e atualizada.
\end{quote}

\section{Desenvolvedores \textit{frontend}}
\label{sec:entrevistasDevs}

De acordo com o roteiro de entrevista ilustrado no Apêndice B deste projeto, os desenvolvedores avaliados foram submetidos a algumas perguntas genéricas onde puderam expor um pouco dos detalhes do seu dia-a-dia de trabalho. O \autoref{table:devsResearch} apresenta as características de cada um dos profissionais entrevistados.

\begin{quadro}
\centering
\begin{tabular}{|m{4cm}|m{10cm}|} \hline
	
	\multicolumn{1}{|c|}{\bfseries Voluntário} & \multicolumn{1}{c|}{\bfseries Contexto} \\\hline
	
	 Desenvolvedor 1 & Profissional que teve mais contato com a base de código-fonte da plataforma. Com mais de 4 anos de colaboração com o produto, teve a oportunidade de participar de vários projetos da empresa. Atualmente, assume o desafio de liderança do \textit{squad} de suporte técnico da organização. \\\hline
	 
	 Desenvolvedor 2 & Principal desenvolvedor do \textit{squad} responsável por manter um dos mais recentes e promissores produtos da empresa. Nesse projeto, teve a oportunidade de trabalhar com tecnologias mais inovadoras do que aquelas utilizadas na base de código-fonte legada da plataforma. \\\hline
	 
	 Desenvolvedor 3 & Mais novo colaborador do \textit{squad} de comunicação da empresa. Está a cerca de 2 meses assumindo o desafio de manter e evoluir sistemas legados de \textit{fronted} da plataforma. \\\hline
    
\end{tabular}
\caption{Características dos desenvolvedores entrevistados}
\fonte{Próprio autor}
\label{table:devsResearch}
\end{quadro}

A seguir, são apresentados os resultados consolidados de perguntas chave realizadas nas entrevistas.

\textbf{Questão 1:} Contextualize, brevemente, a maneira como os processos de desenvolvimento \textit{frontend} são adotados na empresa, atualmente.

\begin{quote}
    Em momentos anteriores à definição dos primeiros processos de design, era papel do desenvolvedor definir e implementar as interfaces de usuário. Não havia nenhuma atividade de pesquisa e validação com os usuários finais.
    
    A partir de então, especificações de páginas web já vêm sendo realizadas por profissionais especializadas e compartilhadas com o desenvolvedor por meio de ferramentas como Sketch e Zeplin. Percebeu-se, como consequência da adoção de tais ferramentas e processo, uma considerável melhoria na velocidade de desevolvimento e da qualidade final das telas.
    
    Após a implementação das interfaces, inicia-se uma fase de validação do artefato em um ambiente controlado chamado de \textit{staging}. Nesse ambiente, o designer certifica-se de que o foi produzido está realmente de acordo com suas especificações.
    
    Caso aprovado, o código-fonte produzido é incorporado à plataforma e começa a gerar valor para o usuário final.
\end{quote}

\textbf{Questão 2:} Quais as maiores dificuldades e frustrações que você encontra na plataforma durante o desenvolvimento de suas atividades?

\begin{quote}
    A maioria das frustrações apresentadas pelos voluntários residem na atual estrutura do principal produto da empresa, conhecido como \textit{dashboard}. Trata-se de um projeto legado, que teve o início de sua construção iniciado à cerca de 7 anos atrás e que utiliza uma \textit{stack} tecnológica atualmente bastante defasada.
    
    Como consequência da sua arquitetura degrada e pela falta de testes de interface no projeto, existe grande receio em se realizar alterações em alguns componentes, pois abre-se a possibilidade de ocorrência de efeitos colaterais inesperados e catastróficos. Por conta disso, é bastante comum encontrarmos diferentes soluções para um mesmo problema. Existe muito código-fonte duplicado ao longo do projeto.
    
    Não existe, ainda, nenhuma biblioteca de componentes bem definida que facilite o reuso de soluções já implementadas e padronize a forma como se constrói novas interfaces de usuário.
\end{quote}

\textbf{Questão 3:} Como você classifica a velocidade de desenvolvimento de novas interfaces de usuários atualmente? Por que?

\begin{quote}
    Está aquém do ideal. Tarefas de desenvolvimento \textit{frontend} são, em geral, bastante custosas de serem realizadas. Atividades que deveriam ser simples de serem executadas, normalmente não são.
    
    É necessário adquirir-se certa fluência com a estrutura e deficiências do projeto. Só assim é possível atingir uma velocidade de desenvolvimento satisfatória. Em geral, o tempo para se atingir tal fluência não é curto, muito por conta da complexidade da base de código-fonte da plataforma e do uso de uma \textit{stack} tecnológica defasada.
\end{quote}

\textbf{Questão 4:} De modo geral, como você classifica a atual arquitetura frontend da plataforma a nível de escalabilidade e reusabilidade?

\begin{quote}
    Levando em consideração a capacidade de escalabilidade do produto, houveram respostas contraditórias, uma vez que as mesmas foram embasadas em perspectivas distintas à respeito da arquitetura.
    
    Em um primeiro momento, foi levantado que a arquitura é escalável, pois o padrão MVC proposto pelo \textit{framework} utilizado na plataforma proporciona tal fenômeno. \textit{Frameworks} que oferecem mais flexibilidade na maneira como se organiza o projeto, podem contribuir para a construção de arquiteturas pouco escaláveis, uma vez que podem não existir padrões bem definidos no projeto como um todo.
    
    Em contrapartida, existe também a visão de que a plataforma é pouco escalável, pois há um forte acoplamento entre o \textit{backend} e o \textit{frontend}. A renderização das interfaces de usuários acontecem, muitas vezes, de maneira mista: parte acontece no universo \textit{frontend} e a outra parte no \textit{backend}. Esse tipo de exemplo de inconsistência é um fator recorrente em toda a plataforma, o que prejudica seu potencial de escalabilidade.
    
    No quesito reusabilidade, o parecer foi unânime. A plataforma é pouco reusável, apresentando várias ocorrências de duplicidade de código-fonte como alternativa para se evitar efeitos colaterais em cenários alternativos.
\end{quote}

\textbf{Questão 5:} Você já ouviu falar no termo \textit{design system}? Se sim, acredita que poderia ser uma abordagem plausível para impulsinar a capacidade de escalabilidade do produto da empresa?

\begin{quote}
    Novamente, de maneira unânime, sim.

    Para se trabalhar em uma plataforma que cresce em velocidade recorde ano após ano, a existência de um \textit{design system} é fator fundamental para atingir as expectativas de velocidade aumejadas pela organização.
    
    Entretando, assim como o observado pelos designers, entende-se que a criação de um \textit{design system} completo é uma tarefa árdua. Necessita-se de profissionais qualificados e que estejam realmente focados em construir uma ferramenta tão ambiciosa assim.
    
    Para se mitigiar o fênomeno da degradação e defasamento das tecnologias adotadas, idealiza-se a contrução de uma bilbioteca de componentes de maneira desacoplada de tecnologias ou frameworks específicos. O ideal seria construir a biblioteca de componentes baseados nos chamados \textit{Web Components}. Dessa forma, migrações de tecnologias defasadas para tecnologias mais recentes acontecerão de maneira menos traumática, uma vez que não existe dependência alguma entre o legado e a biblioteca de componentes em si.
    
    Por fim, também foi apontada a preocupação de se manter as diretrizes expostas pelo \textit{design system} constantemente atualizadas, de forma a se evitar o seu desuso.
\end{quote}

Conforme observado pelas entrevistas, muitos dos problemas apresentados no capítulo 2 deste projeto, também se manifestam na empresa avaliada no experimento. Dada sua atual conjuntura, tal como a realidade do seu produto, a necessidade de criação de um \textit{design system} foi comprovada.

No capítulo seguinte será apresentada a biblioteca de componentes criada, em um formato de protótipo, para se iniciar o processo de surgimento do novo \textit{design system} da organização.
