% -----------------------------------------------------------------------------
% Resumo
% -----------------------------------------------------------------------------

\begin{resumo}
    Reusabilidade e consistência de interfaces de usuário, tanto em termos de design quanto a nível de implementação, vem apresentando problemas desde a popularização das interfaces gráficas de usuário. Bibliotecas de componentes se propuseram a resolver tais problemas, entretanto, devido à vigente segregação entre os domínios do designer e do desenvolvedor, por muito tempo seu objetivo final não foi alcançado. Recentemente um novo conceito chamado \textit{design system} foi proposto como abordagem para a resolução destes problemas. Pretendo avaliar a eficácia e viabilidade do novo paradigma, este trabalho se dedica a realizar um experimento prático em uma empresa de tecnologia de médio porte. O experimento foi dividido em cinco momentos. O primeiro deles destinado à revisão da literatura e de trabalhos relacionado, tanto no âmbito acadêmico quanto no industrial. Em seguida, foram realizadas entrevistas com designers e desenvolvedores \textit{frontend} da empresa avaliada, com o objetivo de comprovar a necessidade de implementação de um \textit{design system}. Seguiu-se com a construção de uma biblioteca de componentes, em um formato de protótipo. O quarto estágio foi destinado à criação de páginas Web embasadas no protótipo de biblioteca de componentes construído anteriormente. Finaliza-se com o detalhamento da avaliação de resultados. De maneira geral, o experimento realizado contribuiu com a ideia de que um \textit{design system} deve ser solução para problemas de escalabilidade e comunicação de um produto.

    \textbf{Palavras-chave}: \textit{Design System}. \textit{Atomic Design}.
\end{resumo}

% -----------------------------------------------------------------------------
% Escolha de 3 a 6 palavras ou termos que descrevam bem o seu trabalho.
% As palavras-chaves são utilizadas para indexação. A letra inicial de cada
% palavra deve estar em maiúsculas. As palavras-chave são separadas por ponto.
% -----------------------------------------------------------------------------
